\documentclass{beamer}
\usetheme[sidebar,rule]{Amurmaple}

% Packages
\usepackage[utf8]{inputenc}
\usepackage[T1]{fontenc}
\usepackage{lmodern}
\usepackage{amsmath}
\usepackage{listings}
\usepackage{tikz}

% Code settings
\lstset{
  basicstyle=\ttfamily\footnotesize,
  backgroundcolor=\color{gray!10},
  frame=single
}

% Metadata
\title{AmurMaple Theme Demo}
\subtitle{Showcase of All Features}
\author{Demo Author}
\institute{Example University}
\date{\today}
\mail{demo@example.com}
\webpage{https://example.com}

\begin{document}

% Title
\frame{\titlepage}

% TOC
\begin{frame}{Table of Contents}
  \tableofcontents
\end{frame}

% Color Variants
\section{Theme Color Variants}

\sepframe

\begin{frame}{Available Color Schemes}
  The AmurMaple theme supports four color variants:

  \begin{itemize}
    \item \textbf{Red} (default): \texttt{\textbackslash usetheme\{Amurmaple\}}
    \item \textbf{Blue}: \texttt{\textbackslash usetheme[amurmapleblue]\{Amurmaple\}}
    \item \textbf{Green}: \texttt{\textbackslash usetheme[amurmaplegreen]\{Amurmaple\}}
    \item \textbf{Black}: \texttt{\textbackslash usetheme[amurmapleblack]\{Amurmaple\}}
  \end{itemize}

  \bigskip

  \begin{block}{Current Theme}
    This presentation uses the default \textcolor{structure}{Red} variant with sidebar and rule options.
  \end{block}
\end{frame}

% Theme Options
\section{Theme Options}

\sepframe

\begin{frame}{Available Options}
  \framesection{Navigation}

  \begin{itemize}
    \item \texttt{sidebar} - Add sidebar navigation (currently active)
    \item \texttt{sidebarwidth=<dim>} - Set sidebar width (default: 58pt)
    \item \texttt{toplogo} - Place logo at top of sidebar
  \end{itemize}

  \framesection{Visual}

  \begin{itemize}
    \item \texttt{rule} - Add decorative rule (currently active)
    \item \texttt{rulecolor=<color>} - Set rule color
    \item \texttt{nogauge} - Disable progress gauge
    \item \texttt{leftframetitle} - Left-align frame titles
  \end{itemize}

  \framesection{Advanced}

  \begin{itemize}
    \item \texttt{delaunay} - Decorative Delaunay triangulation (LuaLaTeX only)
    \item \texttt{nomail} - Disable email display
  \end{itemize}
\end{frame}

% Block Types
\section{Block Types}

\sepframe

\begin{frame}{Standard Blocks}
  \begin{block}{Standard Block}
    This is a standard block for general content.
  \end{block}

  \begin{alertblock}{Alert Block}
    This is an alert block for warnings or important information.
  \end{alertblock}

  \begin{exampleblock}{Example Block}
    This is an example block for code examples or case studies.
  \end{exampleblock}
\end{frame}

\begin{frame}{Mathematical Environments}
  \begin{theorem}[Pythagorean Theorem]
    In a right triangle, $a^2 + b^2 = c^2$ where $c$ is the hypotenuse.
  \end{theorem}

  \begin{definition}[Limit]
    We say $\lim_{x \to a} f(x) = L$ if for every $\varepsilon > 0$,
    there exists $\delta > 0$ such that $|f(x) - L| < \varepsilon$
    whenever $0 < |x - a| < \delta$.
  \end{definition}

  \begin{corollary}
    For any real number $x$, we have $\sin^2(x) + \cos^2(x) = 1$.
  \end{corollary}
\end{frame}

\begin{frame}{Custom Environments}
  \begin{information}[Important Note]
    This is an information box for highlighting key facts.
  \end{information}

  \begin{remark}[Technical Detail]
    This is a remark box for additional context or technical notes.
  \end{remark}
\end{frame}

% Overlays
\section{Overlays and Animations}

\sepframe

\begin{frame}{Step-by-Step Reveals}
  Key points revealed one at a time:

  \begin{itemize}
    \item<1-> First point appears on slide 1
    \item<2-> Second point appears on slide 2
    \item<3-> Third point appears on slide 3
    \item<4-> Fourth point appears on slide 4
  \end{itemize}

  \onslide<5->{
    \begin{block}{Final Note}
      All items are now visible!
    \end{block}
  }
\end{frame}

\begin{frame}{Highlighting with Overlays}
  \begin{itemize}
    \item<1-| alert@1> This item is highlighted on first slide
    \item<2-| alert@2> This item is highlighted on second slide
    \item<3-| alert@3> This item is highlighted on third slide
    \item<4-> This item appears normally
  \end{itemize}
\end{frame}

% Columns
\section{Layouts and Columns}

\sepframe

\begin{frame}{Two-Column Layout}
  \begin{columns}[T]
    \begin{column}{0.48\textwidth}
      \framesection{Left Column}
      \begin{itemize}
        \item First point
        \item Second point
        \item Third point
      \end{itemize}
    \end{column}
    \begin{column}{0.48\textwidth}
      \framesection{Right Column}
      \begin{enumerate}
        \item Numbered item 1
        \item Numbered item 2
        \item Numbered item 3
      \end{enumerate}
    \end{column}
  \end{columns}
\end{frame}

\begin{frame}{Mixed Content Columns}
  \begin{columns}[T]
    \begin{column}{0.48\textwidth}
      \begin{block}{Code Example}
        \begin{lstlisting}[language=Python]
def hello():
    print("Hello!")
        \end{lstlisting}
      \end{block}
    \end{column}
    \begin{column}{0.48\textwidth}
      \begin{exampleblock}{Output}
        \texttt{Hello!}
      \end{exampleblock}

      \bigskip

      Function prints a greeting message to the console.
    \end{column}
  \end{columns}
\end{frame}

% Mathematics
\section{Mathematical Typesetting}

\sepframe

\begin{frame}{Mathematical Equations}
  \framesection{Inline Math}

  The quadratic formula is $x = \frac{-b \pm \sqrt{b^2-4ac}}{2a}$.

  \framesection{Display Math}

  \[
    \int_{-\infty}^{\infty} e^{-x^2} dx = \sqrt{\pi}
  \]

  \framesection{Aligned Equations}

  \begin{align}
    E &= mc^2 \\
    F &= ma \\
    p &= mv
  \end{align}
\end{frame}

\begin{frame}{Matrix Operations}
  \[
    \begin{bmatrix}
      a_{11} & a_{12} & a_{13} \\
      a_{21} & a_{22} & a_{23} \\
      a_{31} & a_{32} & a_{33}
    \end{bmatrix}
    \begin{bmatrix}
      x_1 \\ x_2 \\ x_3
    \end{bmatrix}
    =
    \begin{bmatrix}
      b_1 \\ b_2 \\ b_3
    \end{bmatrix}
  \]

  \bigskip

  Solving systems of linear equations in matrix form.
\end{frame>

% Special Commands
\section{Special Commands}

\sepframe

\begin{frame}{Frame Section Command}
  The \texttt{\textbackslash framesection\{\}} command creates subsections within frames:

  \framesection{First Subsection}
  Content for the first subsection goes here.

  \framesection{Second Subsection}
  Content for the second subsection goes here.

  \framesection{Third Subsection}
  Content for the third subsection goes here.
\end{frame}

\begin{frame}{Box Alert}
  Regular text with \boxalert{highlighted alert text} inline.

  \bigskip

  Use \texttt{\textbackslash boxalert\{\}} for inline emphasis.

  \bigskip

  Multiple uses: \boxalert{Alert 1} and \boxalert{Alert 2} and \boxalert{Alert 3}.
\end{frame}

% Abstract and Quotations
\section{Text Environments}

\sepframe

\begin{frame}{Abstract Environment}
  \begin{abstract}
    This is an abstract environment designed for presentation summaries.
    It features bordered styling that makes it stand out from regular content.
    Use it at the beginning of academic presentations to provide an overview.
  \end{abstract}
\end{frame}

\begin{frame}{Quotation Environment}
  \begin{quotation}[Albert Einstein]
    Imagination is more important than knowledge. For knowledge is limited,
    whereas imagination embraces the entire world, stimulating progress,
    giving birth to evolution.
  \end{quotation}

  \bigskip

  \begin{quotation}
    Anonymous quotations are also supported without the optional author parameter.
  \end{quotation}
\end{frame}

% Lists
\section{List Styles}

\sepframe

\begin{frame}{Itemized Lists}
  \begin{itemize}
    \item First level item
    \item Another first level item
    \begin{itemize}
      \item Second level item
      \item Another second level item
      \begin{itemize}
        \item Third level item
      \end{itemize}
    \end{itemize}
    \item Back to first level
  \end{itemize}
\end{frame}

\begin{frame}{Enumerated Lists}
  \begin{enumerate}
    \item First numbered item
    \item Second numbered item
    \begin{enumerate}
      \item Sub-item a
      \item Sub-item b
      \item Sub-item c
    \end{enumerate}
    \item Third numbered item
  \end{enumerate}
\end{frame}

\begin{frame}{Description Lists}
  \begin{description}
    \item[Term 1] Definition of the first term goes here with details.
    \item[Term 2] Definition of the second term with more information.
    \item[Term 3] Definition of the third term and its explanation.
  \end{description}
\end{frame>

% Code Listings
\section{Code Listings}

\sepframe

\begin{frame}[fragile]{Python Code}
  \begin{lstlisting}[language=Python]
def fibonacci(n):
    """Generate Fibonacci sequence up to n"""
    a, b = 0, 1
    result = []
    while a < n:
        result.append(a)
        a, b = b, a + b
    return result

# Generate first 10 Fibonacci numbers
print(fibonacci(100))
  \end{lstlisting}
\end{frame}

\begin{frame}[fragile]{JavaScript Code}
  \begin{lstlisting}[language=JavaScript]
// React component example
function Counter() {
  const [count, setCount] = useState(0);

  return (
    <div>
      <p>Count: {count}</p>
      <button onClick={() => setCount(count + 1)}>
        Increment
      </button>
    </div>
  );
}
  \end{lstlisting}
\end{frame}

% Graphics
\section{Graphics and TikZ}

\sepframe

\begin{frame}{Simple TikZ Diagram}
  \begin{center}
    \begin{tikzpicture}
      \draw[thick, ->] (0,0) -- (4,0) node[anchor=north] {$x$};
      \draw[thick, ->] (0,0) -- (0,3) node[anchor=east] {$y$};

      \draw[domain=0:3.5, smooth, variable=\x, blue, thick]
        plot ({\x}, {0.5*\x*\x});

      \node[blue] at (2,2.5) {$f(x) = \frac{1}{2}x^2$};
    \end{tikzpicture}
  \end{center}
\end{frame}

% Navigation
\section{Navigation Features}

\sepframe

\begin{frame}{Page Numbers and Progress}
  \begin{itemize}
    \item Page numbers shown at bottom: current/total
    \item Progress gauge in sidebar (if enabled)
    \item Appendix pages numbered separately with Roman numerals
    \item Short title and author in sidebar
  \end{itemize}

  \begin{block}{Navigation Tips}
    \begin{itemize}
      \item Use sidebar for section navigation
      \item Click on section titles to jump to sections
      \item Progress gauge shows completion percentage
    \end{itemize}
  \end{block}
\end{frame}

% Summary
\section{Summary}

\sepframe

\begin{frame}{AmurMaple Features Summary}
  \begin{enumerate}
    \item Four color variants (red, blue, green, black)
    \item Multiple layout options (sidebar, frame title alignment)
    \item Rich block types and environments
    \item Overlay and animation support
    \item Flexible column layouts
    \item Beautiful mathematical typesetting
    \item Code listing integration
    \item Custom commands and environments
    \item Professional navigation
    \item Comprehensive theming options
  \end{enumerate}
\end{frame}

% Thank you
\thanksframe{Thank you!\\ \small Explore the AmurMaple theme for your presentations}

\end{document}
